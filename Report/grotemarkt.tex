% Activate the following line by filling in the right side. If for example the name of the root file is Main.tex, write
% "...root = Main.tex" if the chapter file is in the same directory, and "...root = ../Main.tex" if the chapter is in a subdirectory.
 
%!TEX root =  secondDraft

\chapter[Grote Markt]{Grote Markt}

In the previous chapters, we moved from a non-spatial simulation, to a very simple spatial simulation. In the simple spatial simulations, agents move through an environment, but that environment is very trivial - there are two houses, that is about it. A simulation with non-trivial environment might lead to more uncertainty, and require more interesting agent-behaviours that might prove challenging for the Bayesian Network. Additionally, by implementing non-trivial environments, we move closer to the state-of-the-art agent models for modelling crime cases.

\section{Introduction}


\section{Evaluation}



\subsection{Numbers}
\begin{enumerate}
\item \textbf{Do the conditional frequencies in the BN correspond to the conditional frequencies in the simulation?}
\item \textbf{Could you find the numbers?}
\item \textbf{How robust are the BNs to reduced precision in the CPTs?}
\end{enumerate}

\subsection{Structural}

\begin{enumerate}
\item \textbf{Does the BN represent all relevant events of the scenario?}
\item \textbf{Does the BN have temporal ordering of the hypothesis nodes?}
\item \textbf{Does the BN follow the evidence-idioms (evidence not parents of hypothesis nodes)?}
\item \textbf{Can the BN represent multiple alternative scenarios?}

\end{enumerate}

\subsection{Predictive Skills}
\begin{enumerate}
\item \textbf{How does the BN respond to evidence?}
\item \textbf{When does the BN predict the wrong outcome?}
\end{enumerate}

\section{Discussion}


\subsection{Problems}
\item \textbf{Does the BN not depend on private knowledge?}
\item \textbf{Could we apply this BN to a different location? }


\subsection{Legal Interpretation}

\end{enumerate}

